\documentclass{article}
\usepackage{authblk}
\usepackage[dvips]{graphics}
\usepackage{hyperref}
\hypersetup{
  pdfauthor = {Jeremie Gaidamour, Jonathan J. Hu, Christopher M. Siefert, Ray S. Tuminaro},
  pdftitle = {Accomplishment, Energy Minimizing AMG Algorithms},
  pdfkeywords = {multigrid, algebraic multigrid, AMG, energy minimization},
}

\newcommand{\mcol}{\multicolumn}

\begin{document}
\begin{centering}
{\large Energy-Minimizing Algebraic Multigrid Solvers for Systems of Partial Differential Equations}\\
\vskip 0.1in
J. Gaidamour, J. Hu, C. Siefert, and R. Tuminaro (consultant)\\
\vskip 0.1in
\end{centering}
%
The repeated implicit solution of large sparse linear systems is a key computational challenge
in many parallel calculations at Sandia and often dominates the total simulation time.  The solvers in the Trilinos framework are frequently used
for these systems, in particular the smoothed aggregation (SA) multigrid
methods in the ML library.
%The SA methods are well-developed but can struggle with linear systems arising
%from problems with multiphysics and/or severely stretched meshes, as might be found in in structural mechanics
%or electro-magnetics.  Stretched meshes can reflect either underlying anisotropic phenomena or meshing
%technology limitations.
The SA methods are well-developed but can struggle with linear systems arising
from problems with multiphysics and/or severely stretched meshes, as might be found in in structural mechanics
or electro-magnetics.  Stretched meshes can reflect either underlying anisotropic phenomena or meshing
technology limitations.
In this CSRF project we have developed a family of novel
energy-minimizing multigrid algorithms  that address these shortcomings
while providing faster solution times.
These new algorithms have the flexibility to take any sparsity pattern for transfer
matrices that propagate information within the multigrid hierarchy.  This allows us to choose patterns
tailored for problem physics.
Additionally, the cost of applying these algorithms is lower than SA 
while still achieving accurate interpolation of key problematic error modes.
This yields methods that still have very good convergence properties.
In 3D linear elasticity mesh scaling studies,
energy minimization is always cheaper to apply (meaning faster run times) than SA.  As the mesh stretching
becomes more severe, energy minimization converges faster than SA. 
As an example, the new method is about $2.5$ times faster than SA
on a $40^3$ mesh where the stretching in the $x$ direction is 100 times either the $y$ or
$z$ directions.

\end{document}
