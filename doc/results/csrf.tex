\documentclass{article}
\usepackage{authblk}
\usepackage[dvips]{graphics}
\usepackage{hyperref}
\hypersetup{
  pdfauthor = {Jeremie Gaidamour, Jonathan J. Hu, Christopher M. Siefert, Ray S. Tuminaro},
  pdftitle = {Accomplishment, Energy Minimizing AMG Algorithms},
  pdfkeywords = {multigrid, algebraic multigrid, AMG, energy minimization},
}

\newcommand{\mcol}{\multicolumn}

\begin{document}
\begin{centering}
{\large Energy-Minimizing AMG for PDE Systems}\\
\vskip 0.1in
J. Gaidamour, J. Hu, C. Siefert, and R. Tuminaro (consultant)\\
\vskip 0.1in
\end{centering}
Many Sandia parallel simulations rely on the Trilinos framework for the repeated implicit solution
of linear systems.
One crucial linear solver is ML, whose primary methods are based on smoothed
aggregation (SA) algebraic multigrid.
A successful multigrid solver like ML
accelerates the solution of the linear system of interest
by employing related coarser (smaller) linear systems that
must capture certain near-kernel components of the fine level system.
SA has a number of known weaknesses, however.
First, the cost to apply SA can
grow unacceptably for systems of PDEs and for large number of near-kernel components.
This problem is exacerbated by anisotropic phenomena and higher dimensionality.
Second, SA also has rigid requirements for the sparsity pattern of the matrices that transfer data between
coarse matrices.


In this CSRF project we have developed a family of novel
energy-minimizing coarsening algorithms  that address the aforementioned shortcomings in SA.
These new algorithms have the flexibility to take any sparsity pattern for the transfer
matrices.  This allows us to choose patterns which are tailored for anisotropic problems, for instance.
In addition, the cost of applying these algorithms is lower than SA because the number of
degrees of freedom per grid point does not grow with the number of near-kernel components.
Our algorithms still achieve accurate interpolation of the near nullspace
components while minimizing the energies of the transfer matrices.
A scaling study of
iteration counts and complexities (cost to apply the preconditioner) for SA and energy
minimization is summarized in Table \ref{scaling results}.  The test problem is 3D linear elasticity on
a logically rectangular mesh.  The stretching factor in the $x$ direction is given by $\epsilon$.
For the isotropic case, energy minimization is cheaper to apply (meaning faster run times) than SA but has 
comparable convergence rates.
For the anisotropic cases, energy minimization is cheaper and converges faster than SA.
As an example,
the new method is about $2.5$ times faster than SA for
$\epsilon=100$ on the $40^3$ mesh.
%run time savings.

\newcommand{\spa}{\hspace{0.3cm}}

\begin{table}[b]
\begin{center}
  \begin{tabular}{|c||c@{\spa}c|c@{\spa}c||c@{\spa}c|c@{\spa}c||c@{\spa}c|c@{\spa}c|}
    \hline
    & \mcol{4}{c||}{$\epsilon=1$}  & \mcol{4}{c||}{$\epsilon=10$} & \mcol{4}{c|}{$\epsilon=100$} \\
    \cline{2-13}
    \raisebox{1.5ex}[0cm][0cm]{Mesh}
                 &     \mcol{2}{c|}{SA}  & \mcol{2}{c||}{Emin}  &  \mcol{2}{c|}{SA}  & \mcol{2}{c||}{Emin}  & \mcol{2}{c|}{SA}  & \mcol{2}{c|}{Emin}   \\\hline

%    $5^3$        & 5 & \textit{1.20}   & 5  & \textit{1.05}  & 6  &  \textit{1.76} & 7  & \textit{1.16}  & 6  & \textit{1.82} & 7  & \textit{1.16} \\\hline
    $10^3$       & 6 &  \textit{1.32}   & 7  & \textit{1.10}  & 7  &  \textit{2.05} & 8  & \textit{1.24}  & 8  & \textit{1.98} & 8  & \textit{1.24} \\\hline
    $15^3$       & 8 &  \textit{1.19}   & 8  & \textit{1.05}  & 9  &  \textit{1.69} & 11 & \textit{1.16}  & 10 & \textit{1.68} & 12 & \textit{1.16} \\\hline
    $20^3$       & 8 &  \textit{1.24}   & 8  & \textit{1.06}  & 9  &  \textit{1.82} & 9  & \textit{1.18}  & 11 & \textit{1.76} & 10 & \textit{1.18} \\\hline
    $25^3$       & 9 &  \textit{1.26}   & 9  & \textit{1.08}  & 10 &  \textit{1.98} & 9  & \textit{1.21}  & 12 & \textit{2.02} & 9  & \textit{1.21} \\\hline
    $30^3$       & 10 & \textit{1.22}   & 10 & \textit{1.06}  & 11 &  \textit{1.77} & 11 & \textit{1.17}  & 13 & \textit{1.75} & 13 & \textit{1.17} \\\hline
    $35^3$       & 10 & \textit{1.24}   & 9  & \textit{1.06}  & 11 &  \textit{1.99} & 9  & \textit{1.18}  & 13 & \textit{1.91} & 11 & \textit{1.18} \\\hline
    $40^3$       & 10 & \textit{1.26}   & 10 & \textit{1.07}  & 12 &  \textit{1.83} & 9  & \textit{1.19}  & 16 & \textit{1.77} & 10 & \textit{1.19} \\\hline
  \end{tabular}
\end{center}
\vspace{-0.1cm} % avoid a new page
  \caption{Iteration count and complexity (lower complexity = faster run time) for
  SA and energy minimization for various mesh sizes and stretch factors for
  a 3D linear elasticity problem.}
  \label{scaling results}
\end{table}

\end{document}
