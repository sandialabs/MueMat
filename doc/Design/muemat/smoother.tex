The standard procedure for setting smoothers in the MG hierarchy is to
create a smoother factory and pass this to MgHierarchy.SetSmoothers()
which automatically invokes the smoother factory's build method to
create smoothers on all levels.

\subsection{Previous design: discussion}

This part discuss the initial design of the factories responsible to the creation of smoothers in MueMat.

Smoothers factories are specialized in the production of a particular kind of smoother:
\begin{itemize}
\item A smoother factory stores the parameters of the object they will produce.
\item Smoother parameters depends on the smoother (nits, omega for Jacobi; nits, lambdaratio for Chebychev...). 
\item Factories know the exact class of object that will be created.
\end{itemize}

A smoother factory have a method Build() to create smoother object
depending on the data of the MG level (depending on the matrix A of
the level for instance).  Build() do the setup phase of the
smoother. This step could be costly and must be done only once for pre
and post smoother (for example, the LU factorization for the
DirectSolve smoother). Build() instantiate pre and post smoother,
passing in arguments the data of the setup phase ans the parameters of
the smoother stored in the factory.

A smoother factory produces 2 smoothers at a time (pre and post
smoother). Each factory produces only one kind of smoother.  We can
build completely independent pre and post smoothing objects by
invoking MgHierarchy.SetSmoothers() twice with two different smoother
factories.

Problems:
\begin{itemize}
\item To add a smoother, we must add a factory or modify an existing
factory.
\item Factories stores parameters (nits, omega, forward backward step,
blocking scheme) for both Pre and Post smoother in is class variables.
\item Setup phase of a smoother depends on the smoother. I prefer to
have a Setup() function in a smoother rather than in a factory. Like
this, Setup() and Apply() are on the same file.
\item Each smoother factories built 2 smoothers: a pre-smoother and a
post-smoother.  I was initially thinking to modify this behavior by
storing one pre-smoother factory and one post-smoother factory in the
MG hierarchy but it's actually harder to avoid duplicate setup
computation if smoother factories create only one smoother at a time.
\end{itemize}

\subsection{New design}

The new design is based on the prototype design pattern 
\href{http://en.wikipedia.org/wiki/Prototype_pattern}{[1]}
\href{http://www.oodesign.com/prototype-pattern.html}{[2]}.
%http://en.wikipedia.org/wiki/Factory_method_pattern

My (rstumin) understanding is that a prototype is similar to a factory. 
In the factory pattern, we give parameters to the factory that it stores, but it never 
stores real data. The factory knows how to make objects and 
so when Build() is invoked it creates the desired object using 
the supplied data. It is then ready to build new objects once
new data is supplied. 
In the prototype design pattern, factories use prototypes as model to
make objects. Factories doesn't know anything about the
parameters. The parameters are stored in the prototypes.  The
prototype give also the exact type of object to build (ie:
ChebySmoother, ILUSmoother, etc.). 
The main advantage of prototype pattern is that the factory don't need
to know anything about the object to create. The factory can be very
generic.

A prototype is like an unfinished object and never store real data.
To create a object from the prototype, we do not directly populate 
the data fields of the prototype as this would mean that the prototype would have data
associated with it and so it wouldn't make sense to build another
object from it. To make the desired object, the prototype is instead cloned
and the cloned version's data fields are populated. In this way,
the original prototype remains data-less. The way we use this
for smoothers is as follows. When we want to build a pre-smoother, 
the pre-smoother prototype is cloned and the clone's data fields are 
populated. When we want to build a post-smoother, we can clone the 
post-smoother's prototype and populate the clone's fields. However, if 
the pre- and post-smoother are essentially identical with perhaps 
different parameters which do not affect setup, we can instead clone 
the pre-smoother (which already has data fields set) and copy over the 
parameters from the post-smoother prototype.  The main scenario for this 
is doing something like an ILU smoother where we do not want to perform
the factorization twice. 

%% Factory Pattern:
%% - the factory store parameter
%% - you pass on input of the build method the real data
%% - the factory creates a new object 'smoother'. this object stores both parameters and real data.

%% Prototype Factory:
%% - the factory uses prototyped object as model to create other objects. The factory don't know anything about the parameters. The parameters are stored in the prototypes.
%% - you pass on input of the build method the real data
%% - the factory clones the prototype in order to create a new object 'smoother'. This object get the parameters from the prototype and the data from the input parameters of the build method.

A generic smoother factory is defined. The smoother factory uses
prototypical instances of smoothers and prototypes are cloned to
produce new objects. There is one prototype for the presmoother and
one for the postsmoother. Thus, one factory can produce two different
smoothers for pre and post-smoothing (but still produces both).
Prototypes are stores in the factory.

The type of smoother to create is determined by the prototypical instances.
Prototypes stores they own parameters (nits, omega) and factory don't have any knowledge about that.

\subsubsection{Instantiation of the Smoother Factory}
The 'client' instantiated the smoother factory by passing in argument to the Smoother Factory constructor a pre-configured prototypes.

Ex:

\begin{verbatim}
SmooFactory = SmootherFactory(ChebySmoother(5,1/30))
\end{verbatim}

or 

\begin{verbatim}
nIts        = 5;
lambdaRatio = 1/30;
Smoo        = ChebySmoother()
Smoo        = Smoo.SetIts(nIts);
Smoo        = Smoo.SetLambdaRatio(1/30); 
SmooFactory = SmootherFactory(Smoo);
\end{verbatim}

To use different pre and post smoother, two prototypes can be passed in argument:
\begin{verbatim}
PreSmoo     = ChebySmoother(2, 1/30)
PostSmoo    = ChebySmoother(10,1/30)
SmooFactory = SmootherFactory(PreSmoo, PostSmoo);
\end{verbatim}

[] is a valid smoother which do nothing:
\begin{verbatim}
PostSmoo    = ChebySmoother(10,1/30)
SmooFactory = SmootherFactory([], PostSmoo);
\end{verbatim}

\subsection{Production of Smoother object by the Smoother Factory}
Factory.Build() clones its prototypes and call Setup() on the new objects to create fully functional smoothers for the current hierarchy level. Smoothers do is one setup (ie: DirectSolve compute his LU factorization).
If  pre and post smoother are of the same, Setup() is call only one, on the presmoother. To obtain the post-smoother, the presmoother is cloned and its parameters are changed according to the parameters of the post-smoother.

%Idea of that design pattern: rather than creation it uses cloning. If the cost of creating a new object is large and creation is resource intensive, we clone the object.

% todo: discussion here if setup() depends on parameters.

\subsection{The Smoother class}
A smoother must have the capability to
\begin{itemize}
\item Clone itself
\item Copy is parameters
\item + Setup() and Apply()
\end{itemize}

%an abstract base class that specifies a pure virtual clone() method.
% any class that needs a "polymorphic constructor" capability derives itself from the abstract base class, and implements the clone() operation.
%If needed, possibility to keep the same interface, by calling the factory method with a parameter designating the particular concrete derived class desired

\subsection{Special case of Hybrid2x2 Smoother}

The Hybrid2x2 Smoother could be created by the generic SmootherFactory but in this case, we are not able to avoid duplicate setup computation. A special factory (Hybrid2x2SmootherFactory) must be used on this case. 
The factory create two temporary smoother factories to do the setup of Pre and Post Smoother of each part of the system at the same time.

