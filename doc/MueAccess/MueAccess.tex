\documentclass{siamltex}

\advance\oddsidemargin by -0.25in
\advance\evensidemargin by -0.25in
\advance\textwidth by .55in

\usepackage{amsfonts,amssymb,amsmath}
%\usepackage{subfigure}
\usepackage[dvips,pdftex]{graphicx}
\usepackage{subfig}
\usepackage{multirow}
\usepackage{multicol}
\usepackage{url}
\usepackage{calc,array}
\usepackage{booktabs}
\usepackage[usenames,dvipsnames]{color}
\usepackage[linesnumbered,noline,ruled,noend]{}
\newtheorem{thm}{Theorem}[section]
\newtheorem{lem}[thm]{Lemma}
\newtheorem{rem}[thm]{Remark}
\DeclareMathOperator*{\argmin}{argmin}
\newcommand{\Bcoarse}{\ensuremath{B^{C}}}
\newcommand{\nns}{near null-space}
\DeclareGraphicsRule{.pdftex}{pdf}{*}{}
\usepackage{algorithmic}
\usepackage{algorithm}
\numberwithin{algorithm}{section}  % <--- chapter, section etc. depending on what is required
\def\Boxers#1#2{\noindent$\hphantom{Muv}${\parbox[t]{1.75in}{\it
#1}}{\parbox[t]{2.9in}{#2}} \\[0.7em]}

\title{Level Interface: getting at level data in MueMat/MueLu}

\author{J.~Hu
\and    J.~Gaidamour
\and    R.~Tuminaro
\and    T.~Wiesner}

% For making comments in the document.
\newcommand{\JJH}[1]{\textcolor{red}{JJH: #1}}
\newcommand{\JG}[1]{\textcolor{cyan}{JG: #1}}
\newcommand{\RST}[1]{\textcolor{Blue}{RST: #1}}
\newcommand{\TW}[1]{\textcolor{RoyalBlue}{TW: #1}}


\begin{document}
\maketitle


\section{Test Cases} \label{test cases}
Some cases to consider (note: auxiliary matrix and Emin(Emin) examples are interesting but would not be used heavily).
\vskip .1in
\begin{enumerate}
\item SA with coalesced $A$ for aggregation.\\

\item SA with filtered then coalesced $A$ for aggregation \& filtered $A$ for
   prolongator smoother.\\

\item SA with user-defined auxiliary graph for aggregation \& a decoalesced
   version to define matrix for prolongator smoother.\\

\item SA with user-defined auxiliary graph for aggregation \&
   prolongator smoother matrix is a filtered $A$ where pattern
   matches decoalesced version of auxiliary matrix.
   \newcounter{enumi_saved}
    \setcounter{enumi_saved}{\value{enumi}}
\end{enumerate}
\vskip .1in
%Emin algorithms are trickier. The main additional things to play with include:
%\vskip .1in
%\Boxers{P0ForPattern}{obvious}
%\Boxers{P0ForKrylovInitialGuess}{obvious}
%\Boxers{P0ForAPpattern}{obvious}
%\Boxers{AForAPpattern}{obvious}
%\Boxers{AForEnergy}{obvious}
%\Boxers{MForKrylov}{might want to consider preconditioning the Krylov 
%minimizer with arbitrary block diagonals.}
%\Boxers{FNull,CNull,Constraints}{Only MakeConstraints need FNull and CNull.
%                              While FNull/CNull might be defined in different
%                              ways, I'm not sure that we would ever have a
%                              valid case where we would want more than one
%                              type of null space on the same level. We might
%                              someday want to consider constraints other than
%                              P CNull = FNull.  I'm thinking mostly of
%                              something where we only care about adding
%                              constraints locally to enforce local features
%                              like preserving a discontinuity.}
\begin{enumerate}
      \setcounter{enumi}{\value{enumi_saved}}
\item Standard Emin using an initial guess of $Ptent$ (obtained by aggregating
   coalesced A).\\
\item Badri-like case where two user-provided matrices are each filtered
      to provide different matrices for Emin: one is Krylov
      energy \& the other for use by APpattern.\\
\item Emin using 1-step SA as initial guess.\\

\item Emin using Emin as initial guess where each computes a Ptent and Coarse Null space \& both are to be saved for later reuse.\\

\item Convert an SA factory so that it can be used for restriction.\\

\item A test case for an F/C-style sparsity pattern (e.g,
      F/C pattern generates a P0)?

\end{enumerate}
\section{Function Pointer Interface}
\subsection{LevelInterface class}
A simple class with only two things: static functions and data member 
function pointers The basic idea is that the function pointers are used in
specific factories to access data. They effectively define the factory's
interface to Level.  They are set to one of the particular static functions. 
Developers add new static functions as needed and perhaps new function 
pointers as well.  There is no attempt to group functions into subclasses
or to hide some functions from some factories.  The static functions
are extremely basic and are not meant to hide complex logic. They are
strictly for data access.  A sample matlab-style class is shown below
\vskip .1in
\begin{verbatim}
classdef LevelInterface < CopiableHandle

  properties (Access = public)

    % function pointers (null or set to a static function) defining interface.
    
   P0ForPattern            = 
   P0ForKrylovInitialGuess = @LevelInterface.GetPtent
   P0ForAPpattern          = @LevelInterface.GetPtent
   AForAPpattern           = @LevelInterface.GetA
   AForEnergy              = @LevelInterface.GetA
   GraphForCoarsening      = @LevelInterface.GetGraphOfA
   NullForPtent            = @LevelInterface.GetNull
   NullForConstraints      = @LevelInterface.GetNull

   SetFilteredA            = @LevelInterface.SetFilteredA
   SetGraph                = @LevelInterface.SetGraph
   SetCoalescedA           = @LevelInterface.SetCoalescedA
   SetNull                 = @LevelInterface.SetNull
   MForKrylov
   Constraints
   CPoints

  end %properties

  methods (Static = true)

    % static functions for use by the above function ptrs

    function [A] = GetA(Level)
        A = Level.GetA();               end 
    function [A] = GetATrans(Level)
        A = Level.CheckOut("ATrans");   end
    function [A] = GetFilteredA(Level)  
        A = Level.CheckOut("Afiltered");end
    function [A] = GetFilteredAAux(Level)  
        A = Level.CheckOut("AfiltAux"); end
    function [A] = GetCoalescedA(Level)
        A = Level.CheckOut("ACoalesce");end
    function [P] = GetP(Level)        
        P = Level.GetP();               end
    function [P] = GetPtent(Level)    
        P = Level.CheckOut("Ptent");    end
    function [P] = GetPAux(Level) 
        P = Level.CheckOut("PAux");     end 
    function [G] = GetGraphOfA(Level) 
        G = Level.CheckOut("Graph");    end
    function [N] = GetNull(Level)     
        N = Level.CheckOut("Null");     end
    function [N] = GetNullAux(Level)     
        N = Level.CheckOut("NullAux");  end
    function [ ] = SetA(Level, A)          
        Level.SetA();                   end
    function [ ] = SetFilteredA(Level,A)    
        Level.Save("Afiltered",A);      end
    function [ ] = SetFilteredAAux(Level,A)    
        Level.Save("AfiltAux",A);       end
    function [ ] = SetCoalescedA(Level,A)  
        Level.Save("ACoalesce",A);      end
    function [ ] = SetNull(Level,Null)          
        Level.Save("Null",Null);        end
    function [ ] = SetP(Level,P)          
        Level.SetP();                   end 
    function [ ] = SetPtent(Level,P)      
        Level.Save("Ptent",P);          end
    function [ ] = SetPAux(Level,P)   
        Level.Save("PAux",P);           end
    function [ ] = SetGraphOfA(Level,G)
       Level.Save("Graph",G);           end

  end % methods (Static = true)

end %Operator classdef
\end{verbatim}
\vskip .1in
Note: CheckOut() and Save() are MueLu ideas to be incorporated in MueMat.
To simplify the presentation, data members are assumed public
(things can obviously done via Set/Get). 

\subsection{Factory Access}
One can make arguments in favor of a Level-centric view or a Factory-centric
view. Here, a bit of both is proposed where each factory makes the final 
decision.  In particular, each Level has a field which is of type 
LevelInterface called LevelsView.  Inheritance would be highly discouraged
for this LevelsView.  When users create a level, they can do things like
\vskip .1in
\begin{verbatim}
    FineLevel.LevelsView.AForAPpattern = @GetFilteredA;
    FineLevel.LevelsView.AForEnergy    = @GetFilteredA;
\end{verbatim}
\vskip .1in
Again, Set/Get functions can be used to hide things.  When a new coarse level 
is created by Populate(), it copies FineLevel.LevelsView.  In this way, all 
levels have a well-defined interface. If the user wants different behavior 
on a coarser level, he can manually create a coarse level and set its
LevelsView function pointers to whatever he wants.

So far, this looks Level-centric, but we can allow each factory to decide
what it does with the interface. For example, a factory could have a private
member of type LevelInterface called FactorysView.  The factory's build() 
now has two LevelInterfaces: FactorysView and LevelsView.  The build() 
implementer decides how these are merged to create a consistent interface 
called MergedView that is used within the method. Inheritance is permitted in
FactorysView and MergedView.  As a merge example, consider 
\begin{itemize}
\item function pointers in FactorysView are NULL except for a couple 
      where behavior is to be altered in a factory specific way.
\item build() copies LevelsView to MergedView (which is local to 
      build()).  It then copies all nonNull function pointers 
      from FactorysView to MergedView. 
\item MergedView is now used exclusively within build().
\end{itemize}

This is meant as a simple example to give the idea. Factories could
do powerful (but perhaps dangerous things) if they wanted. For example,
suppose that build() not only wants to change how it accesses
data but how other build methods in other factories that it calls access the 
Level. It could save LevelsView, then copy MergedView into the Level structure
(overwriting LevelsView), and finally just before exiting the
build function it could restore the saved copy back in the Level. If
inheritance is discouraged in LevelsView, then only an uninherited version
of the MergedView is stored in the Level.  There are other possibilities. 
Another merge idea (mimicking Jeremie's view of different SmartLayers 
building on each other) is a capability to define a MergedView function 
pointer as a copy of a DIFFERENT LevelsView function pointer.  For example,
SA by definition uses the same matrix to define both energy and the AP pattern.
One way to recognize this in the interface is via
\vskip .1in
\begin{verbatim}
     MergedView.AForAPpattern = LevelsView.AForAPpattern. 
     MergedView.AForEnergy    = LevelsView.AForAPpattern. 
\end{verbatim}
\vskip .1in
That is, MergedView is just a reshuffling of function pointers in 
LevelsView. This effectively defines a mapping from one interface to
a new one. There are also other ambitious possibilities. For example, a 
factory could use the Save/CheckOut mechanism to store additional 
interfaces within a level's data.  So long as another factory knows what 
to look for, it can pull out these other interfaces and use them when
forming its MergedView.  

\subsection{Strengths}
\begin{itemize}
\item Handles important cases.
\item Easy to understand without looking at much documentation.
\item Factory implementers have a lot of freedom while allowing a
      basic mechanism to associate an interface with a level.
\item Could probably be changed easily if we decide later to adopt another
      model (Tobias' Communication Layer or Jeremie's SmartLevel).
\end{itemize}
\subsection{Weakness}
\begin{itemize}
\item Does not logically group or organize access functions.
\item Factories could do dangerous things.
\item \TW{The \textit{LevelInterface} class emulates the old-style \textit{Level} class interface. You may need auxiliary variables in the \textit{InterfaceLevel} class (e.g. \textit{GetPAux}, for test case 7). If we have a highly flexible \textit{Level} class with a hashtable, it doesn't make sense for me to introduce an interface class for the \textit{Level} class that uses predefined auxiliary matrices like \textit{GetPAux}. In my opinion we should try to make use of the \textit{Level} class and not hide it behind an interface like \textit{LevelInterface}.}
\item \TW{not object-oriented $\rightarrow$ port to MueLu?}
\item Bring it on ... I can take it!
\end{itemize}


\subsection{Implementation of Test Cases}
\begin{enumerate}
\item SA with coalesced A for aggregation.
   \begin{verbatim}
   Standard SA case requiring only defaults.
   \end{verbatim}
\item SA with filtered then coalesced $A$ for aggregation \& filtered $A$ for
   prolongator smoother.\\
   \begin{verbatim}
   FineLevel.AForAPpattern = @LevelInterface.GetFilteredA;
   FineLevel.AForEnergy    = @LevelInterface.GetFilteredA;
   \end{verbatim}
\item SA with user-defined auxiliary matrix for aggregation \& a decoalesced
   version to define matrix for prolongator smoother.\\
   \begin{verbatim}
   FineLevel.AForAPpattern      = @LevelInterface.GetFilteredA
   FineLevel.AForEnergy         = @LevelInterface.GetFilteredA

   This strange case can be addressed by a version of CoalesceAndDrop
   which decoalesce graphs. Instead of polluting LevelsView, we make an
   extended version of FactorysView with function pointers
           GetGraphForDecoalescing = @LevelInterface.GetGraphOfA
           SetDecoalescedA         = @LevelInterface.SetFilteredA.
   The user puts the Graph in the right fine level place and provides
   code to project graphs to coarse levels.
   \end{verbatim}
\item SA with user-defined auxiliary graph for aggregation \&
   prolongator smoother matrix is a filtered $A$ where pattern
   matches decoalesced version of auxiliary matrix.
   \begin{verbatim}
   Same as above. Special CoalesceAndDrop decoalesces, but does not
   store decoalesced matrix. Instead it is used to filter A and store 
   result in standard filtered spot(thus, no need for SetDecoalescedA)
   \end{verbatim}
\item Standard Emin using an initial guess of $Ptent$ (obtained by aggregating
   coalesced A).
   \begin{verbatim}
   Just use defaults.
   \end{verbatim}
\item Badri-like case where two user-provided matrices are each filtered
      to provide different matrices for Emin: one is Krylov
      energy \& the other for use by APpattern.
   \begin{verbatim}
   FineLevel.AForAPpattern      = @LevelInterface.GetFilteredA
   FineLevel.AForEnergy         = @LevelInterface.GetFilteredAAux

   The standard CoalesceAndDrop factory is extended so that it can 
   filter two different matrices and store them. Inheritance is
   used to extend FactorsView so that it has functions like
       GetMatrixForFilteredEnergy,  SetMatrixForFilteredEnergy 
       GetMatrixForFilteredPattern, SetMatrixForFilteredPattern
   \end{verbatim}
\item Emin using 1-step SA as initial guess.
   \begin{verbatim}
      SAFactorysView.SetP           = @LevelInterface.SetPAux
      EminFactorysView.P0ForPattern = @LevelInterface.GetPAux
      EminFactorysView.P0ForKrylovInitialGuess= @LevelInterface.GetPAux
   \end{verbatim}
\item Emin using Emin as initial guess where each computes a Ptent and Coarse Null space \& both are to be saved for later reuse.
   \begin{verbatim}
   Create two Emin factories: OuterEmin & InnerEmin. InnerEmin is
   standard. OuterEmin is changed so that it Sets/Gets the Ptent
   and CNull from nonstandard spots.
      OuterEminFactorysView.GetPtent = @LevelInterface.GetPAux
      OuterEminFactorysView.SetPtent = @LevelInterface.SetPAux
      OuterEminFactorysView.GetNull  = @LevelInterface.GetNullAux
      OuterEminFactorysView.SetNull  = @LevelInterface.SetNullAux
   Also modify Needs/CrossFactory string so counters are properly set?
   \end{verbatim}
\item Convert an SA factory so that it can be used for restriction.
   \begin{verbatim}
   Extend SA factory so that it computes/stores A^T, computes/stores
   Rtent^T, and invokes standard build() with
        FactorysView.GetPtent      = @LevelInterace.GetTransRtent
        FactorysView.AForAPpattern = @LevelInterface.GetATrans
        FactoriysView.AForEnergy   = @LevelInterface.GetATrans
        FactorysView.SetPtent      = @LevelInterace.SetTransRtent
   \end{verbatim}
\item A test case for an F/C-style sparsity pattern (e.g,
      F/C pattern generates a P0)?
\end{enumerate}

\section{Communication layer}

\subsection{CommunicationLayer class}
Every factory class needs some input information and produces some output. For example, the \verb|SA-AMG| algorithm takes an initial prolongation operator (e.g. the tentative prolongation operator) as input and calculates a smoothed prolongation operator. The idea is that all \verb|Build| functions in the factory classes use a standardized set of interface functions for the communication with the data source in the \verb|Level| class. In the simple case of smoothed aggregation (\verb|SA-AMG|), the \verb|SaPFactory.BuildP()| function would call something like \verb|GetInitialP()| and \verb|GetA()| for the initial prolongation operator and the level matrix $A$ for smoothing. These standardized interface functions are handled by the communication layer. Of course we could put all these interface functions into the \verb|PFactory| class, but i think it's better to introduce an own class with interface functions to be more flexible.

A sample matlab-style class for a DefaultCommunicationLayer is shown below. The intention of such a DefaultCommunicationLayer class is to provide a standard implementation for the most common interface functions, that are used/needed by most factory classes.
\begin{verbatim}
classdef DefaultCommunicationLayer < CopiableHandle
% default implementation for transfer operator communication layer 

    properties (Access=protected)
        % not sure if we need the following variables: 
        % the idea is that every factory class has some input and 
        % some output:
        % [Level/user/?] -> input variables -> [Factory class] ...
        %                           -> output variables -> [Level]
        % most simple case: the user prescribes which variable is
        % to be used for input and what variable name is to be used
        % for storing the results.
        % For more complicated test cases (e.g. 3
        % and 4, you'll derive from this DefaultCommunicationLayer
        % class a new specialized communication layer class, that
        % you can extend for your needs.
        % Note: if we store a pointer of the generating factory
        % we always can use "default names" for the  variables.
        % That should make things much easier! Then we also could
        % use static member functions, that is, there's nearly no
        % difference to the LevelInterface class (as proposed by Ray).
        PinitialFromVariable_ = []; % variable name for initial P (input)
        RinitialFromVariable_ = []; % variable name for initial R (input)
                                    % (i suppose that we don't need this, 
                                    % we always can use the transposed
                                    % of variable PinitialFromVariable_)
        storePas_ = [];             % variable name that shall be used
                                    % for storing result of the factory
                                    % class (prolongator)
        storeRas_ = [];             % variable name that shall be used
                                    % for storing result of the factory
                                    % class (restrictor)
    end

    methods
        %% constructor for default communication layer
        % if not prescribed by the user, set default variable names for input and output variables.
        function [this] = TransferOpCommunication(fromPInitial,toP,fromRinitial,toR)
            if nargin == 1 && isa(fromPInitial, class(this)), this.Copy_(fromPInitial,[]); return; end
            
            if ~varexist('fromPInitial') this.PinitialFromVariable_ = 'Pinitial';
            else this.PinitialFromVariable_ = fromPInitial; end;

            if ~varexist('fromRInitial') this.RinitialFromVariable_ = 'Rinitial';
            else this.RinitialFromVariable_ = fromRInitial; end;

            
            if ~varexist('toP') this.storePas_ = 'P';
            else this.storePas_ = toP; end;

            if ~varexist('toR') this.storeRas_ = 'R';
            else this.storeRas_ = toR; end;

        end
        
        %% get/set functions for input/output variable names

        % we should get rid of these functions and store factory
        % pointers together with the default variable 
        % names (as Jeremie suggested).
        function [oldPvariable] = StorePas(this,Pvar)
            if varexist('Pvar') 
                oldPvariable = this.storePas_;
                this.storePas_ = Pvar;
            else
                oldPvariable = this.storePas_;
            end
        end

        function [oldRvariable] = StoreRas(this,Rvar)
            if varexist('Rvar') 
                oldRvariable = this.storeRas_;
                this.storeRas_ = Rvar;
            else
                oldRvariable = this.storeRas_;
            end
        end
        
        function [oldPinitialvar] = PinitialFrom(this,Pvar)
            if varexist('Pvar') 
                oldPinitialvar = this.PinitialFromVariable_;
                this.PinitialFromVariable_ = Pvar;
            else
                oldPinitialvar = this.PinitialFromVariable_;
            end
        end

        function [oldRinitialvar] = RinitialFrom(this,Rvar)
            if varexist('Rvar') 
                oldRinitialvar = this.RinitialFromVariable_;
                this.RinitialFromVariable_ = Rvar;
            else
                oldRinitialvar = this.RinitialFromVariable_;
            end
        end
        
        %% communication layer level interface functions

        % The following functions correspond to the static member functions
        % in the LevelInterface class (proposed by Ray)
        % Here, i have only three of these level interface functions as example. 
        % I suppose we need some more, but the idea is to have only a as small 
        % as possible set of common functions that is used by most/all 
        % (P)Factory classes. 
        % For more special [P]Factory classes (e.g. Emin) i would rather
        % introduce a EminCommunicationLayer class, that derives from 
        % DefaultCommunicationLayer class, than add some functions like
        % GetPAux() or things like AForEnergy, that you only need in 
        % some special cases.

        % In your Factory.Build functions you have to call these 
        % "standardized" level interface functions, that are defined
        % and handled here by the communication layer.
        % The parameter list for such level interface functions should 
        % always contain the FineLevel and CoarseLevel objects, that are
        % used for the communication with the level class.

        % GetInitialP provides an initial prolongation operator and the
        % corresponding coarse grid nullspacace. Either use a user
        % prescribed variable name, or generate a tentative
        % prolongation operator.
        % TODO: move prolongation_mode_ flag from PFactory to CommunicationLayer...
        function [P,cnull] = GetInitialP(this, prolongation_mode, FineLevel,...
                                                           CoarseLevel, Specs)
         if(CoarseLevel.IsData(this.PinitialFromVariable_))
                if isempty(CoarseLevel.GetNull())
                    error('no carse level stored in CoarseLevel\n');
                end
                P = CoarseLevel.Get(this.PinitialFromVariable_);
                cnull = CoarseLevel.GetNull();
          else

                % use TentativePFactory with default parameters 
                % note: generate a default communication layer class for the
                % tentative prolongation PFactory
                % set correct input/output variable names (no input for tentative
                % prolongation necessary)
                PinitialLevelCommunicator = DefaultCommunicationLayer(...
                   [],this.PinitialFromVariable_,[],this.RinitialFromVariable_); 
                   % communication flow: [] -> this.PinitialFromVariable_ (='Pinitial')
                TentPFact = TentativePFactory(); 
                TentPFact.SetLevelCommunicator(PinitialLevelCommunicator);

                % overwrites CoarseLevel.GetNull()
                TentPFact.BuildP(FineLevel,CoarseLevel,Specs);

                P = CoarseLevel.Get(this.PinitialFromVariable_);
                cnull = CoarseLevel.GetNull();
          end
      end  % end GetInitialP

      function [A] = GetA(this, prolongation_mode, ...
                                    FineLevel, CoarseLevel, Specs);
      %GETA
      %
      %   SYNTAX   [A] = GetA(prolongation_mode, FineLevel, CoarseLevel, Specs);
      %
      %     prolongation_mode - mode of communication object (true or false)
      %     FineLevel         - Level object for fine level
      %     CoarseLevel       - Level object for coarse level
      %     A                 - level system matrix for prolongation smoothing
      %
      % default implementation for PFactory derived prolongation operators
      % with support of "restriction" mode
      % The prolongation_mode flag controls what is used as system matrix
      % for smoothing the prolongation operator: A in "prolongation" mode
      % and the transposed of A in the "restriction" mode
          if prolongation_mode == true
              % PFactory is in prolongation mode
              % use system matrix A
              A = FineLevel.GetA();
          else
              % PFactory is in restriction mode
              % use the transposed of A (downwinding)
              A = FineLevel.GetA()';
          end
      end   % end GetA
      
      function SetTransferOperator(this, prolongation_mode, FineLevel, ...
                                     CoarseLevel, Specs, TransferOperator)
      %SETTRANSFEROPERATOR
      %
      %   SYNTAX   SetTransferOperator(CoarseLevel, TransferOperator;
      %
      %     prolongation_mode- denotes if PFactory is in prolongation or
      %                        restriction mode
      %     FineLevel        - Level object for fine level
      %     CoarseLevel      - Level object for coarse level
      %     Specs            - ...
      %     TransferOperator - prolongator (or transposed of restrictor in
      %                        "restriction" mode)
      %
      % stores the result of transfer operator smoothing in level data
      % structure as prolongator (if prolongation_mode_==true) or as
      % restrictor (if prolongation_mode_==false).
      % This is the default implementation for prolongation operator with
      % support of "restriction mode".
          if prolongation_mode == true
              % PFactory is in prolongation mode
              % set prolongation operator
              CoarseLevel.Set(this.storePas_,TransferOperator);
          else
              % PFactory is in restriction mode
              % set restriction operator
              CoarseLevel.Set(this.storeRas_,TransferOperator');
          end
      end      
    end
    
    methods (Access = protected)
        % for copy constructor
        function Copy_(this, src, mc)
          %COPY_
          %
          %   SYNTAX   obj.Copy_(src, mc);
          %
          %     src - Object to copy
          %     mc  - MATLAB Metaclass
          [cmd, data, mc] = this.CopyCmd_(src,mc);
          eval(cmd);
        end        
    end

end % end classdef
\end{verbatim}

\subsection{Factory Access}

Let me give you a small and simple example, how the communication layer class is used within a PFactory class.
\begin{verbatim}
 classdef PgPFactory < PFactory

  properties (Access = private)
      % this PFactory class supports a prolongation and restriction mode
      % default behaviour: prolongation mode
      % if prolongation_mode_ is set to false, the BuildP function is used
      % for generating a restriction operator (using A^T -> communication layer)
      % TODO: move me to communication layer
      prolongation_mode_ = true;

      initialPFact_         % PFactory for initial guess of prolongator
      ...
  end

  methods   % public functions
      %% constructor
      function [this] = PgPFactory(initialPFact,diagonalView) 

          % constructor sets options

          % initialize inter-factory communication
          % default data flow: 'Pinitial' -> 'P', 'Rinitial' -> 'R'
          this.myLevelComm_ = TransferOpCommunication();

          % set PFactory for initial guess
          if varexist('initialPFact'), 
              this.initialPFact_ = initialPFact; 
          end;

          % set diagonal view ?
          if varexist('diagonalView'), this.diagonalView_ = diagonalView; end
      end


      function [ToF] = ProlongationMode(this, ToF)
          if varexist('ToF'),
              ToFold = this.prolongation_mode_;
              this.prolongation_mode_ = ToF;
              ToF = ToFold;
          else ToF = this.prolongation_mode_; end
      end

.....

      %% build prolongator
      function flag = BuildP(this,FineLevel,CoarseLevel,Specs)
          % construct PG-AMG prolongator
          flag = true;

          this.TempOutputLevel(Specs);

          %% 1) check for system matrix and prepare fine level nullspace
          % get system matrix from level storage 
          % (depending on prolongation_mode_)
          Amat = this.myLevelComm_.GetA(this.prolongation_mode_,...
                   FineLevel,CoarseLevel,Specs); % use communication layer
          NS   = FineLevel.Get('cnull');% fine level NS (note: direct
                                        % communication with Level is
                                        % also possible!)

          ...

          %% 2) build initial prolongation operator (based on aggregates)

          % build initial prolongator 
          % (data flow: [] -> CoarseLevel.'Pinitial')
          % either the user has to make sure, that the data flow
          % is set correctly or this is done automatically (should
          % be easy if we store not only the variable names, but
          % also the factory, which generated the data. Then we can
          % "link" the Pinitial_ factory with the current PFactory class)
          if ~isempty(this.initialPFact_)
               this.initialPFact_.BuildP(FineLevel, CoarseLevel,Specs);  
          end;

          % use communication layer
          [P,cnull] = this.myLevelComm_.GetInitialP(...
                  this.prolongation_mode_, FineLevel,CoarseLevel,Specs);

          % PGAMG smoothing...

          %% 10) store PG-AMG transfer operator in CoarseLevel level class
          % use communication layer 
          % store transfer operator (data flow: P -> 'P' or 'R',
          % depending on prolongation_mode_)
          this.myLevelComm_.SetTransferOperator(this.prolongation_mode_, ...
                                            FineLevel, CoarseLevel, Specs,P);

          this.RestoreOutputLevel();
      end
  end   %  end public methods
  ...
end
\end{verbatim}

There are some differences to the LevelInterface class, Ray proposed, that i want to point out here:
\begin{itemize}
 \item Ray uses only one (common) LevelView class with a sepearate instance of the LevelView class for every multigrid level object. Optionally every factory class can have its own instance of the LevelView class (FactoryView) that is then merged together with the LevelViews of the level class to a MergedView LevelView instance that is used for communication with the level object in the factory Build function. In the contrary to that, i have only one instance of a (Default)Communication layer for every factory class. The \verb|DefaultCommunicationLayer| class, as shown in the previous subsection, corresponds to the LevelView instance of Ray's LevelInterface class. If you need some special communication, you would introduce a new CommunicationLayer class (e.g. EminCommunicationLayer $<$ DefaultCommunicationLayer) and overwrite/add communication functions in there. This \verb|EminCommunicationLayer| then is used as default communication layer class in the \verb|EminPFactory| class. Deriving from \verb|DefaultCommunicationLayer| and adding new communication functions corresponds to Ray's FactoryView concept with merging down LevelView and FactoryView.
 \item Ray proposes, that every level class has its own instance of a \verb|LevelInterface| class. From the \verb|TwoLevelFactoryBase| point of view then you have a FineLevel.LevelView and a CoarseLevel.LevelView. I suppose you have two FactoryViews and two MergedViews, too (one for the fine level, one for the coarse level).
My communication layer classes have always the FineLevel object as well as the coarse level object as parameters. The communication layer functions know, which level object they have to use (e.g. GetA() $\rightarrow$ FineLevel, SetP() $\rightarrow$ CoarseLevel). I think this is a detail we can hide from the ``BuildP()'' developer, since it's only a matter of communication which level object (fine or coarse) is used to store the results.
 \item The idea of the communication layer class is to group some of these communication layer functions in a reasonable way. I think i would even introduce separate communication layer classes for different types of factories. For example pattern factories have some different communication functions than prolongation factories. Why to pollute the communication layer class for PFactory with pattern stuff?
 \item Maybe i'm wrong, but somehow i have the feeling that the \verb|LevelInterface| class tries to ``emulate'' the old style level class (before introducing the hash table, $\rightarrow$ SetAuxP functions, etc.). The communication layer is not just a shell around the level class. It's a layer between the level and the factory class that is adapted to the special needs of the algorithms in the corresponding factory.
\end{itemize}

\subsection{Strengths}
\begin{itemize}
 \item handles important cases ($\rightarrow$ Jeremie's suggestion for extending the \verb|Level| class by storing the generating factory).
 \item communication layer concept is designed for factory developers: as a factory developer you only have to provide the necessary communication layer functions. You can either add them to the factory class (not recommended), to existing communication class with default implementations for the most basic communication layer functions, or even use your own communication layer class for grouping all necessary communication layer functions. For a factory developer this may be somewhat easier to understand, than to change/adapt some interface class to the level class or the level class itself.
 \item it's object-oriented.
 \item default communication functions are logically grouped in default communication layer classes.
 \item You can develop advanced communication layer functions within your communication layer classes that allow to automatically link different factories together (Ptent $\rightarrow$ SaPFactory $\rightarrow$ Emin $\rightarrow$ Emin$\ldots$). You don't have to link some function pointers by hand.
 \item The communication layer functions don't hide the \verb|Level| class behind an interface. They try to make use of the functionality of \verb|Level| for the factory classes.
\end{itemize}

\subsection{Weakness}
\begin{itemize}
 \item Factories could do dangerous things. But we can try to provide very smart default implementations to hide most of the communication stuff from the not so advanced factory developer. So we can minimize the risk that something goes wrong.
 \item If we have some local variables within our communication layer classes (e.g. \verb|prolongation_mode_|, variable names for input/output?) we cannot use static functions. The resulting code for calling communication layer functions from the factories doesn't look very elegant:
\begin{verbatim}
this.myComm_.GetA(FineLevel,CoarseLevel,Specs,maybe even more parameters?);
\end{verbatim}
\item feel free to add your concerns
\end{itemize}

\subsection{Implementation of Test Cases}
\begin{enumerate}
 \item \textit{SA with coalesced A for aggregation:} Standard SA case requiring only defaults. (same for PG-AMG, see above).
 \item \textit{SA with filtered then coalesced A for aggregation \& filtered A for prolongator smoother:} In my opinion it's not so important for the SA smoothing algorithm to know, if it's a filtered system matrix or not. The SA smoothing algorithm just calls \verb|GetA| and gets the correct (filtered or not) matrix for smoothing. Since using a filtered matrix should be a quite common case we should extend the \verb|GetA| function of the \verb|DefaultCommunicationLayer| class for filtered A matrices, too. One of the simplest possibilities would be, that the user first generates the filtered matrix A and stores it in the level matrix (using a variable name, e.g. 'myFilteredA') and adapt the communication layer function \verb|GetA| for user given variable names (similar to what is done for \verb|GetInitialP|). 

There are several possibilities for the pattern: We could add a \verb|GetAforAPpattern| communication function to the \verb|DefaultCommunicationLayer| class, that is only called within the pattern factories. Then we can use the \verb|DefaultCommunicationLayer| class for the \verb|AP_PatternFactory| (and other PatternFactories), too. I would feel better if we have an own \verb|DefaultPatternCommunicationLayer| class with default communication routines for all pattern factories. The \verb|PatternFactory| class is independent from the \verb|PFactory| with its own communication functions. I'm not sure, but i think sometimes we might have different filtered matrices for pattern and smoothing?

Anyway, we can extend the default communication layer classes to support this kind of examples.
 
\item \textit{SA with user-defined auxiliary graph for aggregation \& a decoalesced version to define matrix for prolongator smoother:} Depending how ``strange'' we rank this test case, we have different possibilities to add support for such things:
\begin{enumerate}
\item If it's something that we need very often, we can extend the \verb|DefaultCommunicationClass| by a \verb|GetGraphForDecoalescing| communication function, that is called within the \verb|CoalesceAndDropFactory| class. Then \verb|CoalesceAndDropFactory| uses the same \verb|DefaultCommunicatoinClass| as PFactory classes.
\item Maybe it's even worth to introduce an own communication class for \verb|CoalesceAndDropFactory| classes, that derives from \verb|DefaultCommunicationClass|. In my opinion that's not necessary, but of course we can do it and keep the \verb|DefaultCommunicationClass| clean from stuff, that's only interesting for decoalescing.
\item Note, that if this test case is too strange, so that we don't want to introduce an own communication layer class or pollute the \verb|DefaultCommunicationClass| with such things, it's also ok, if we put the communication layer functions in the factory class itself! Then the \verb|CoalesceAndDropFactory| class has its own (private?) \verb|GetGraphForDecoalescing| function, which communicates with the fine and coarse level classes. The main idea of the communication classes is to put and group some of these communication functions together and reuse them, wherever it's possible and makes sense.
\end{enumerate}
\item \textit{SA with user-defined auxiliary graph for aggregation \& prolongator smoother matrix is a filtered A where pattern matches decoalesced version of auxiliary matrix:} Same as above. I think/hope that the new Level class (semi-) automatically knows which data has to be stored and can be freed.
\item \textit{Standard Emin using an initial guess of \textit{Ptent} (obtained by aggregating coalesced A):} just use defaults.
\item \textit{Badri-like case where two user-provided matrices are each filtered to provide different matrices for Emin: one is Krylov energy \& the other for use by APpattern:} The \verb|GetA(forAPpattern?)| function in the communication layer for pattern factory should handle the (filtered) matrix for APpattern (per default). I think i would introduce an extended \verb|EminCommunicationLayer| class (derived from \verb|DefaultCommunicationLayer|) that handles the special needs of Emin (something like \verb|AForEnergy|, again with support for filtered matrices). In that extended Emin communication class we also can add some additional communication functions that are needed by the CoalesceAndDrop factory. Within the Emin.BuildP function then we have to replace the standard communication layer class for the CoalesceAndDropFactory class with our EminCommunicationLayer... You see, we're very flexible. We only have to make sure, that all factories get their own communication layer class, that fits all their needs. However we should try to provide standard implementations for the most common problems, that can be easily adapted.
\item \textit{Emin using 1-step SA as initial guess:} I hope the new Level class supports something like CoarseLevel.Get('P',Ptent), as Jeremie suggested in one of his mails last week. Then you don't need 'SetPAux'...
\item \textit{Emin using Emin as initial guess where each computes a Ptent and Coarse null space \& both are to be saved for later reuse:} Two different Emin methods in a chain with independent Ptent and Coarse null space should be no problem with Jeremie's proposed level class. Since you're using two different instances of the \verb|TentativePFactory| class you can store it as two ``different'' variables with the same name 'Ptent' in the level class. There should be no need for functions like ``GetPAux''...
\item \textit{Convert an SA factory so that it can be used for restriction:} see above default implementation: i used a \verb|prolongation_mode_| flag within the PgPFactory and MinDescentPFactory class. We should move it to the communication layer, since the algorithms (BuildP()) should be independent of the ``prolongation'' or ``restriction'' mode. The only thing that is changing is the communication layer, that is $A \rightarrow A^T$ and store the result as R instead of P. The \verb|GenericRFactory| class only switches the \verb|prolongation_mode_| flag in the communication class for the corresponding \verb|Pfactory| class, then calls \verb|BuildP| for calculating the restrictor, and finally switches back the \verb|prolongation_mode_| flag to standard (=prolongation) mode.
\item \textit{A test case for an F/C-style sparsity pattern (e.g. F/C pattern generates a P0)?} The pattern specific communication functions should handle the communication between level classes and pattern factory.
\end{enumerate}


\end{document}
